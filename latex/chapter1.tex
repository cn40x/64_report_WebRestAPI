%%================================================
%% Chapter 1
%%================================================
\chapter{บทนำ}
%\label{intro}
\label{chapter1}

\section{ที่มาและความสำคัญ}
ในปัจจุบันปฏิเสธไม่ได้เลยว่าเทคโนโลยีนั้นมีบทบาทสำคัญและได้เข้ามาเป็นส่วนหนึ่งในชีวิตประจำวันของมนุษย์ทุกคนอย่างหลีกเลี่ยงไม่ได้ มนุษย์สามารถเข้าถึงคอมพิวเตอร์ได้มากขึ้น เป็นผลมาจากการมาถึงของอินเทอร์เน็ตและการเปลี่ยนแปลงของโลก  เช่น สถานการณ์การแพร่ระบาดของไวรัสโควิด 19 ทำให้การใช้เทคโนโลยีและอินเทอร์เน็ตนั้นกลายเป็นสภาพแวดล้อมใหม่ในการเรียนและการทำงาน บทบาทที่เป็นสิ่งสำคัญในชีวิตประจำวันของมนุษย์ใน\mbox{หลาย ๆ} ด้าน ดังนี้
\begin{itemize}
    \item การใช้อินเทอร์เน็ตเพื่อความบันเทิง โดยอินเทอร์เน็ตเป็นแหล่งบันเทิงที่รวมความบันเทิงหลากหลายประเภทให้อยู่ในรูปแบบพกพาได้ เช่น สามารถดูโทรทัศน์ ฟังวิทยุ ดูภาพยนตร์ หรืออ่านหนังสือบนอินเทอร์เน็ต ที่สามารถเข้าถึงหรือใช้งานได้ทุกเวลาที่ต้องการ 
    \item การใช้อินเทอร์เน็ตเพื่อการทำธุรกิจและการพาณิชย์ ในยุคปัจจุบัน อินเทอร์เน็ตสามารถนับได้ว่าเป็นแหล่งซื้อขายหรือตลาดขนาดใหญ่ที่สุดในโลก เนื่องจากผู้คนสามารถทำการซื้อขายกันด้วยความสะดวกและรวดเร็วผ่านอินเทอร์เน็ต
    \item การใช้อินเทอร์เน็ตเพื่อการศึกษา หรือการใช้อินเทอร์เน็ตเพื่อการสนับสนุนการศึกษา เป็นการใช้อินเทอร์เน็ตในการติดต่อสื่อสารเพื่อการส่งการบ้าน ส่งข้อสอบ นัดหมาย อธิบายรายละเอียด รวมทั้งแลกเปลี่ยนความคิดเห็นของผู้สอนกับผู้เรียน และผู้เรียนกับผู้เรียนด้วยกัน นอกจากนี้ยังมีระบบการเรียนการสอนด้วยระบบออนไลน์ การเรียนออนไลน์ค่อนข้างเป็นที่นิยมมากใน\mbox{ปัจจุบัน} เพราะเป็นการเรียนที่สะดวกสามารถเรียนได้ในเวลาที่ต้องการ อีกทั้งสามารถลดค่าใช้จ่ายในการเรียนลงไปได้มาก และทั้งนี้อินเทอร์เน็ตนั้นยังสามารถใช้เป็นแหล่งค้นคว้าหาข้อมูลเสมือนเป็นห้องสมุดขนาดใหญ่พร้อมทั้งมีข้อมูลที่ครอบคลุมมีทั้งภาพ เสียง และภาพเคลื่อนไหว
\end{itemize}
จากความสำคัญของอินเทอร์เน็ตข้างต้น สามารถเล็งเห็นได้ถึงข้อดีของอินเทอร์เน็ต และยังรวมถึงข้อเสียด้วยเช่นกัน เช่น การใช้อินเทอร์เน็ตเพื่อการทำธุรกิจและการพาณิชย์ หมายถึงการซื้อขายที่ง่ายขึ้น แต่ก็อาจะมีความเสี่ยงในการถูกมิจฉาชีพหลอกหลวงได้ง่ายด้วยเช่นกัน หรือการใช้งานอินเทอร์เน็ตเพื่อการศึกษาก็มีข้อเสีย คือ การที่ผู้สอนไม่ได้มีปฏิสัมพันธ์กับผู้เรียนโดยตรง หรือการที่ผู้เรียนมีสิ่งเร้าจากสภาพแวดล้อมภายนอกคอยรบกวนสมาธิได้ และไม่สามารถให้ความสนใจในบทเรียนหรือผู้สอนมากเท่าที่ควร
จากสาเหตุข้างต้น จึงเป็นที่มาของโครงงาน REST API สำหรับการส่งงานเขียนโปรแกรมด้วย OpenAPI(REST API for Programming Assignment With OpenAPI) ที่จะยกระดับการทำการบ้านหรือการทำข้อสอบประเภทการเขียนโปรแกรมในรูปแบบออนไลน์ ให้มีความสะดวกมากยิ่งขึ้น สามารถตรวจสอบความถูกต้องได้อย่างชัดเจน ป้องกันไม่ให้เกิดการตรวจข้อสอบที่ผิดพลาด และสามารถลดเวลาในการตรวจข้อสอบได้เป็นอย่างมาก
\section{วัตถุประสงค์}
\begin{enumerate}
    \item เพื่อพัฒนา Web Application สําหรับการส่งงานเขียนโปรแกรม
    \item เพื่อเพิ่มความรวดเร็วในการส่งงานหรือส่งข้อสอบการเขียนโปรแกรม
    \item เพื่ออํานวยความสะดวกให้นักศึกษาในการส่งงานให้อาจารย์ผู้สอน
    \item เพื่ออํานวยความสะดวกให้ผู้สอนสำหรับการตรวจงานการเขียนโปรแกรม
    \item เพื่อเพิ่มความแม่นยำและถูกต้องในการตรวจงานการเขียนโปรแกรม
\end{enumerate}

\section{ขอบเขตการดำเนินงาน}
\begin{enumerate}
    \item สร้าง Website ที่อาจารย์ผู้สอนและนักศึกษาสามารถใช้งานได้ตรงตามจุดประสงค์ ซึ่งแบ่งออกเป็น 2 ส่วนดังนี้
\begin{enumerate}[\theenumi.\arabic*]
    \item การใช้งาน Website ส่วนของอาจารย์ผู้สอนในการสร้างวิชาเรียนสำหรับการสร้าง Assignment สำหรับนักศึกษาในการส่งงาน และแสดงคะแนนของนักศึกษา
    \item การใช้งาน Website ส่วนของนักศึกษาในการส่งงานและแสดงคะแนนของนักศึกษา
\end{enumerate}    
    \item สามารถเรียก API ที่ใช้ในการตรวจงานเขียนโปรแกรมของนักศึกษาได้ 
\end{enumerate}
\section{ขั้นตอนการดำเนินงาน}
\begin{enumerate}
    \item กำหนด requirement ของเว็บที่จะทำ
    \item ศึกษาเครื่องมือต่างๆและการทำงานของ Django
    \item ศึกษาวิธีการเขียน Python
    \item ศึกษาวิธีการทำงานของ Docker ในการเรียนใช้ API
    \item ออกแบบหน้า UI ทั้งหมดในการใช้งานจริง
    \item เขียนโปรแกรมในส่วนของหน้า Register และ Login
    \item เขียนโปรแกรมหน้า home page สามารถแสดงรายวิชาต่าง ๆ
    \item เขียนโปรแกรมในส่วนของอาจารย์สามารถแก้ไขหรือเพิ่ม วิชาเรียน และ assignment ได้
    \item เขียนโปรแกรมการแสดงผลคะแนนหลังจากการส่งงานเสร็จ
    \item ทดสอบและแก้ไขข้อผิดพลาด
    \item จัดทำรายงานโครงงานฉบับสมบูรณ์
\end{enumerate}
\section{ผลที่คาดว่าจะได้รับ}
\begin{enumerate}
    \item สามารถเรียกใช้ API สำหรับตรวจงานได้จริง
    \item ผู้สอนมีความสะดวกในการตรวจงานการเขียนโปรแกรม
    \item สามารถนำโครงงานนี้ไปใช้ในการเรียนกการสอนจริง
\end{enumerate}

\begin{landscape}
\section{ตารางการดำเนินงาน}
\begin{table}[h!]
 \centering
 \begin{threeparttable}
  \begin{tabular}{|l|c|c|c|c|c|c|c|c|c|c|c|c|c|c|c|c|c|c|c|c|c|}
  
    \hline
    \multirow{2}{*}{\textbf{แผนการดำเนินงาน}}
    & \multicolumn{4}{c|}{สิงหาคม} 
    & \multicolumn{4}{c|}{กันยายน} 
    & \multicolumn{4}{c|}{ตุลาคม}
    & \multicolumn{4}{c|}{พฤศจิกายน}
    & \multicolumn{4}{c|}{ธันวาคม}\\
    \cline{2-21}
    & 1 & 2 & 3 & 4
    & 1 & 2 & 3 & 4
    & 1 & 2 & 3 & 4
    & 1 & 2 & 3 & 4
    & 1 & 2 & 3 & 4\\
    \hline
    
    
    list requirement ของเว็บที่จะทำ ควรมีอะไรบ้าง &&&x&x&&&&&&&&&&&&&&&&\\
     \hline
    ศึกษาเครื่องมือต่างๆและการทำงานของ Django &&&&&x&x&&&&&&&&&&&&&&\\
    \hline
    ศึกษาวิธีการเขียน Python เพื่อพัฒนา Web Application &&&&&&&x&x&x&x&&&&&&&&&&\\
    \hline
    ออกแบบหน้า UI ทั้งหมดในการใช้งานจริง &&&&&&&&&&&x&x&&&&&&&&\\
    \hline
    เขียนโปรแกรมในส่วนของหน้า Register และ Login &&&&&&&&&&&&&x&x&x&x&&&&\\
    \hline
    เขียนโปรแกรมหน้าhome page และ แสดงรายวิชาต่างๆ &&&&&&&&&&&&&&&&&x&x&x&x\\
    \hline
    
  \end{tabular}
  \caption{การดำเนินโครงงาน}
  \label{table:table}
  \end{threeparttable}
%\hrulefill
\end{table}
%\end{landscape}
%\newpage
%%%%%%%%%%%%%%%%%%%%%%%%%%%%%%%%%%%%%%%%%%%%%%%%%%%%%%%%
%\begin{landscape}
\begin{table}[h!]
 \centering
 \begin{threeparttable}
  \begin{tabular}{|l|c|c|c|c|c|c|c|c|c|c|c|c|c|c|c|c|c|c|c|c|c|}
  
    \hline
    \multirow{2}{*}{\textbf{แผนการดำเนินงาน}}
    & \multicolumn{4}{c|}{มกราคม} 
    & \multicolumn{4}{c|}{กุมภาพันธ์} 
    & \multicolumn{4}{c|}{มีนาคม}
    & \multicolumn{4}{c|}{เมษายน}
    & \multicolumn{4}{c|}{พฤษภาคม}\\
    \cline{2-21}
    & 1 & 2 & 3 & 4
    & 1 & 2 & 3 & 4
    & 1 & 2 & 3 & 4
    & 1 & 2 & 3 & 4
    & 1 & 2 & 3 & 4\\
	\hline
    เขียนโปรแกรมส่วนของผู้สอนสามารถแก้ไขหรือเพิ่ม assignment &x&x&&&&&&&&&&&&&&&&&&\\
    \hline
    เขียนโปรแกรมส่วนของหน้าต่างๆที่ User สามารถใช้งานได้ &&&x&x&x&x&x&x&&&&&&&&&&&&\\
    \hline
    เขียนโปรแกรมการแสดงผลคะแนนหลังจากการ submit งาน &&&&&&&&&x&x&&&&&&&&&&\\    
    \hline
    ทดสอบทำการเรียกใช้ APIและแก้ไขข้อผิดพลาด &&&&&&&&&&&x&x&x&x&&&&&&\\
    \hline 
    จัดทำรายงานโครงงานฉบับสมบูรณ์ &&&&&&&&&&&&&&x&x&x&&&&\\
    \hline
   
  \end{tabular}
  \caption{การดำเนินโครงงาน ( ต่อ )}
  \end{threeparttable}
  \label{table:table2}
  
  
  
%  \hrulefill
\end{table}
\end{landscape}