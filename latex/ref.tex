\chapter{การอ้างอิง}

ระบบอ้างอิงในรูปแบบของมหาวิทยาลัยธรรมศาสตร์ มีอยู่  3 รูปแบบ คือ
\begin{enumerate}[leftmargin=0pt,itemindent=0.6in]
\item แบบ Vancouver (ซึ่งเป็นรูปแบบที่ ใช้ใน template นี้ )
\item แบบ Turabian 
\item แบบ APA
\end{enumerate}

\section{Vancouver}
ซึ่งเป็นรูปแบบที่ตั้งไว้ใน template นี้ โดย คำสั่งที่กำหนดไว้ ใน thesis.tex คือ 
\begin{enumerate}[leftmargin=0pt,itemindent=0.6in]
{\tt
\item  \lstinline|\usepackage[superscript]{cite}| 
 
\item  \lstinline|\bibliographystyle{vancouver}|
\item  \lstinline|\bibliography{refs}|} โดย refs คือ Bibtex file 


\end{enumerate}

\section{APA}
จำเป็นที่ต้องปิดคำสั่ง แบบ Vancouver ก่อน และเพิ่มคำสั่ง {\tt \lstinline|\usepackage{apacite}| }
และ {\tt \lstinline|\bibliographystyle{apacite}|}  แทน เวลา cite ใช้ {\tt \lstinline|\cite{key}|},  {\tt \lstinline|\citeauthor{key}|},
\\ และ {\tt \lstinline|\citeyear{key}|} เป็นต้น

\section{TURABIAN}
พบว่ามีปัญหา ไม่สามารถสร้างการอ้างอิงแบบเชิงอรรถได้ 


